\documentclass[paper=a4, fontsize=11pt, parskip=full]{scrartcl} % A4 paper and 11pt font size

\usepackage[T1]{fontenc} % Use 8-bit encoding that has 256 glyphs
\usepackage{fourier} % Use the Adobe Utopia font for the document - comment this line to return to the LaTeX default
\usepackage[english]{babel} % English language/hyphenation
\usepackage{amsmath,amsfonts,amsthm} % Math packages

\usepackage{graphicx}

\usepackage{float}

%Preamble
\usepackage{listings}
\usepackage{color}
\definecolor{javared}{rgb}{0.6,0,0} % for strings
\definecolor{javagreen}{rgb}{0.25,0.5,0.35} % comments
\definecolor{javapurple}{rgb}{0.5,0,0.35} % keywords
\definecolor{javadocblue}{rgb}{0.25,0.35,0.75} % javadoc

\lstset{language=Java,
basicstyle=\ttfamily,
keywordstyle=\color{javapurple}\bfseries,
stringstyle=\color{javared},
commentstyle=\color{javagreen},
morecomment=[s][\color{javadocblue}]{/**}{*/},
numbers=left,
numberstyle=\tiny\color{black},
stepnumber=2,
numbersep=10pt,
tabsize=4,
showspaces=false,
showstringspaces=false}


\usepackage{hyperref}
\hypersetup{
    colorlinks=true,
    linkcolor=blue,
    filecolor=magenta,
    urlcolor=cyan,
}

\usepackage{lipsum} % Used for inserting dummy 'Lorem ipsum' text into the template

\usepackage{sectsty} % Allows customizing section commands
\allsectionsfont{\centering \normalfont\scshape} % Make all sections centered, the default font and small caps

\usepackage{fancyhdr} % Custom headers and footers
\pagestyle{fancyplain} % Makes all pages in the document conform to the custom headers and footers
\fancyhead{} % No page header - if you want one, create it in the same way as the footers below
\fancyfoot[L]{} % Empty left footer
\fancyfoot[C]{} % Empty center footer
\fancyfoot[R]{\thepage} % Page numbering for right footer
\renewcommand{\headrulewidth}{0pt} % Remove header underlines
\renewcommand{\footrulewidth}{0pt} % Remove footer underlines
\setlength{\headheight}{13.6pt} % Customize the height of the header

\numberwithin{equation}{section} % Number equations within sections (i.e. 1.1, 1.2, 2.1, 2.2 instead of 1, 2, 3, 4)
\numberwithin{figure}{section} % Number figures within sections (i.e. 1.1, 1.2, 2.1, 2.2 instead of 1, 2, 3, 4)
\numberwithin{table}{section} % Number tables within sections (i.e. 1.1, 1.2, 2.1, 2.2 instead of 1, 2, 3, 4)

\setlength\parindent{0pt} % Removes all indentation from paragraphs - comment this line for an assignment with lots of text

%----------------------------------------------------------------------------------------
%	TITLE SECTION
%----------------------------------------------------------------------------------------

\newcommand{\horrule}[1]{\rule{\linewidth}{#1}} % Create horizontal rule command with 1 argument of height

\title{
\normalfont \normalsize
\textsc{University of Virginia, Department of Computer Science} \\ [25pt] % Your university, school and/or department name(s)
\horrule{0.5pt} \\[0.4cm] % Thin top horizontal rule
\huge AVL - Tree Implementations \\ % The assignment title
\horrule{2pt} \\[0.5cm] % Thick bottom horizontal rule
}

\author{Nada Basit and Mark Floryan}

\date{\normalsize\today} % Today's date or a custom date

\begin{document}

\maketitle % Print the title

%----------------------------------------------------------------------------------------
%	Summary
%----------------------------------------------------------------------------------------

\section{Summary}

For this homework, you will be extending your Binary Search Tree to include tree rotations, and self-balancing. You will implement a few methods to complete this AVL Tree.

\begin{enumerate}
	\item Grab your working code from the \emph{Binary Search Trees} assignment (the project / starter code is the same).
	\item Implement the missing methods in the AVLTree class (some of this is done for you to simplify the assignment)
	\item Use the provided tester files to verify your implementation works. Note that you should test your code more so than the provided tester does this time. The tester is NOT as thorough as in previous homeworks.
	\item \textbf{FILES TO DOWNLOAD:} \emph{None, but use your code from the previous assignment for this one.}
	\item \textbf{FILES TO SUBMIT:} BinaryTree.java (from last week), BinarySearchTree.java (from last week), AVLTree.java (new)
\end{enumerate}

%------------------------------------------------

\subsection{AVLTree.java}

You will implement an AVL tree that inherits from your binary search tree. An AVL tree can take advantage of the insert and remove methods from the class it inherits from (i.e., BinarySearchTree.java). Thus, to insert into an avl tree, you can call super.insert() and then simply check if the current node needs to be balanced. Some of this implementation is provided for you, but you will have to implement the following methods yourself:

\begin{lstlisting}
public class AVLTree<T extends Comparable<T>>
					extends BinarySearchTree<T>{

	//Insert and remove
	protected TreeNode<T> insert(T data, TreeNode<T> curNode);
	protected TreeNode<T> remove(T data, TreeNode<T> curNode);


	//figures out whether a double or single rotation is
	//needed and in which direction(s)
	private TreeNode<T> balance(TreeNode<T> curNode);
			
	//rotate right on the curNode provided
	private TreeNode<T> rotateRight(TreeNode<T> curNode);
	
	//rotate left on the curNode provided
	private TreeNode<T> rotateLeft(TreeNode<T> curNode);
	
	//compute the balance factor of the given node
	private int balanceFactor(TreeNode<T> node);

}
\end{lstlisting}


\subsection{Testing your code}

Once you are done, you can look at the two provided tester files to check your implementation. As stated earlier, these files do not rigorously check your implementations, so you should be writing your own test cases in addition to the few provided.

You should submit \textbf{three files} for this homework: \textbf{BinaryTree.java}, \textbf{BinarySearchTree.java}, and \textbf{AVLTree.java}. The former two are resubmissions of your work from last week.

\subsection{Gradescope}

You should submit your code to \emph{Gradescope}. If you are having trouble with your submission, you should double check the following common problems:

\begin{enumerate}
	\item Make sure you are only submitting the three requested files, and they are named \emph{BinaryTree.java}, \emph{BinarySearchTree.java}, and \emph{AVLTree.java} exactly.
	\item Make sure you keep any \emph{package} statements in your code before submitting. The autograder expects your files to have the package statements that are provided in the downloaded project.
	\item Make sure your output is in the correct format. You should not be printing ANYTHING else or the autograder will think your output is incorrect.
\end{enumerate}

%------------------------------------------------


%----------------------------------------------------------------------------------------

\end{document}


%----------------------------------------------------------------------------------------
%----------------------------------------------------------------------------------------
%----------------------------------------------------------------------------------------
%----------------------------------------------------------------------------------------
%----------------------------------------------------------------------------------------
%----------------------------------------------------------------------------------------


%WORKS CITED:

%%%%%%%%%%%%%%%%%%%%%%%%%%%%%%%%%%%%%%%%%
% Short Sectioned Assignment
% LaTeX Template
% Version 1.0 (5/5/12)
%
% This template has been downloaded from:
% http://www.LaTeXTemplates.com
%
% Original author:
% Frits Wenneker (http://www.howtotex.com)
%
% License:
% CC BY-NC-SA 3.0 (http://creativecommons.org/licenses/by-nc-sa/3.0/)
%
%%%%%%%%%%%%%%%%%%%%%%%%%%%%%%%%%%%%%%%%%

%----------------------------------------------------------------------------------------
%	PACKAGES AND OTHER DOCUMENT CONFIGURATIONS
%----------------------------------------------------------------------------------------
