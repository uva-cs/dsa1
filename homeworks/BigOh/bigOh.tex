\documentclass[paper=a4, fontsize=11pt, parskip=full]{scrartcl} % A4 paper and 11pt font size

\usepackage[T1]{fontenc} % Use 8-bit encoding that has 256 glyphs
\usepackage{fourier} % Use the Adobe Utopia font for the document - comment this line to return to the LaTeX default
\usepackage[english]{babel} % English language/hyphenation
\usepackage{amsmath,amsfonts,amsthm} % Math packages

\usepackage{graphicx}

\usepackage{float}

%Preamble
\usepackage{listings}
\usepackage{color}
\definecolor{javared}{rgb}{0.6,0,0} % for strings
\definecolor{javagreen}{rgb}{0.25,0.5,0.35} % comments
\definecolor{javapurple}{rgb}{0.5,0,0.35} % keywords
\definecolor{javadocblue}{rgb}{0.25,0.35,0.75} % javadoc

\lstset{language=Java,
basicstyle=\ttfamily,
keywordstyle=\color{javapurple}\bfseries,
stringstyle=\color{javared},
commentstyle=\color{javagreen},
morecomment=[s][\color{javadocblue}]{/**}{*/},
numbers=left,
numberstyle=\tiny\color{black},
stepnumber=2,
numbersep=10pt,
tabsize=4,
showspaces=false,
showstringspaces=false}


\usepackage{hyperref}
\hypersetup{
    colorlinks=true,
    linkcolor=blue,
    filecolor=magenta,
    urlcolor=cyan,
}

\usepackage{lipsum} % Used for inserting dummy 'Lorem ipsum' text into the template

\usepackage{sectsty} % Allows customizing section commands
\allsectionsfont{\centering \normalfont\scshape} % Make all sections centered, the default font and small caps

\usepackage{fancyhdr} % Custom headers and footers
\pagestyle{fancyplain} % Makes all pages in the document conform to the custom headers and footers
\fancyhead{} % No page header - if you want one, create it in the same way as the footers below
\fancyfoot[L]{} % Empty left footer
\fancyfoot[C]{} % Empty center footer
\fancyfoot[R]{\thepage} % Page numbering for right footer
\renewcommand{\headrulewidth}{0pt} % Remove header underlines
\renewcommand{\footrulewidth}{0pt} % Remove footer underlines
\setlength{\headheight}{13.6pt} % Customize the height of the header

\numberwithin{equation}{section} % Number equations within sections (i.e. 1.1, 1.2, 2.1, 2.2 instead of 1, 2, 3, 4)
\numberwithin{figure}{section} % Number figures within sections (i.e. 1.1, 1.2, 2.1, 2.2 instead of 1, 2, 3, 4)
\numberwithin{table}{section} % Number tables within sections (i.e. 1.1, 1.2, 2.1, 2.2 instead of 1, 2, 3, 4)

\setlength\parindent{0pt} % Removes all indentation from paragraphs - comment this line for an assignment with lots of text

%----------------------------------------------------------------------------------------
%	TITLE SECTION
%----------------------------------------------------------------------------------------

\newcommand{\horrule}[1]{\rule{\linewidth}{#1}} % Create horizontal rule command with 1 argument of height

\title{
\normalfont \normalsize
\textsc{University of Virginia, Department of Computer Science} \\ [25pt] % Your university, school and/or department name(s)
\horrule{0.5pt} \\[0.4cm] % Thin top horizontal rule
\huge Big Oh - Analyzing Scalability of Algorithms \\ % The assignment title
\horrule{2pt} \\[0.5cm] % Thick bottom horizontal rule
}

\author{Nada Basit and Mark Floryan}

\date{\normalsize\today} % Today's date or a custom date

\begin{document}

\maketitle % Print the title

%----------------------------------------------------------------------------------------
%	Scalability of Algorithms
%----------------------------------------------------------------------------------------

\section{Summary}

For this homework, you will implement a few simple methods. Each of these methods is expected to run at a different \emph{asymptotic complexity}, including $\Theta(n)$, $\Theta(\log n)$, $\Theta(n \log n)$, $\Theta(n^2)$, $\Theta(n^3)$, and $\Theta(2^n)$. You will focus on implementing the methods in this homework, and analyze their time complexity for the analysis homework (the next homework).

\begin{enumerate}
	\item Download the starter code and import the project into Eclipse.
	\item Implement the missing methods from BigOh.java
	\item Verify the correctness of your algorithms and submit to Gradescope to check.
	\item \textbf{FILES TO DOWNLOAD:} \href{https://uva-cs.github.io/dsa1/homeworks/BigOh/code/BigOh.zip}{BigOh.zip}
	\item \textbf{FILES TO SUBMIT:} BigOh.java
\end{enumerate}

%------------------------------------------------

\subsection{Download Starter Code and Implement Five Methods}

You can download the starter code for this homework from the course repository \href{https://uva-cs.github.io/dsa1/homeworks/BigOh/code/BigOh.zip}{here}. Once you have done so, you should import the project into Eclipse.

This project contains two Java files. \emph{BigOh.java} contains the methods you need to implement and \emph{Main.java} contains a main function we have written to help you test your code and measure the time each method takes to execute. \emph{BigOh.java} contains the following methods (most of which are not yet implemented):

\begin{lstlisting}
/* Binary Search: Should run in Theta(logn) time */
/* Returns true if item is in the array a */
public static binarySearch(int[] a, int item);

/* Max value in array: Should run in Theta(n) time */
public static int max(int[] a);

/* Calls binary search n times. Counts number of successful searches */
/* You should search for the numbers 1 through n in succession */
/* Should run in Theta(nlogn) time */
public static int multipleBinarySearch(int[] a);

/* Counts pairs of numbers whose sum is multiple of 5 */
/* Should run in Theta(n^2) time */
public static int allPairs(int[] a);

/* Counts the pairs of three in the list a,b,c in which a+b=c */
/* Should run in Theta(n^3) time */
public static int allTriads(int[] a);

/* Prints all subsets of a */
/* Should run in Theta(2^n) time */
/* e..g, {1,2,3} would print {},{1},{2},{3},{1,2},{1,3},{2,3},{1,2,3} */
public static int allSubsets(int[] a);
\end{lstlisting}

Your first task is to implement the first five of these methods. The allSubsets() method is provided as it is much more tricky than the others.

Many of you have probably not seen \textbf{binary search} before (we may go over this algorithm in lecture if we have time). \emph{Binary search} is an efficient method of searching through sorted data. Because the data is sorted, we can efficiently search by looking in the middle of the data and reducing our search to the half that is guaranteed to contain the item. We do this continuously until the search narrows all the way down to one item. This is similar to looking up a word efficiently in a dictionary. First, we look in the middle of the book. If our word is in the lower half we open halfway in that direction, and so on until we narrow in on the word we are searching for. Some psuedo-code for binary search can be found below:

\begin{lstlisting}
binarySearch(int[] a, int item){
	set variables min=0 and max=a.length-1
	continue while min is less than max
		middle = middle element between min and max
		if middle element is item, then found!
		if item is less than middle element
			repeat search between min and middle
		if item is more than middle element
			repeat search between middle+1 and max

	if this loop exits, then the item was not found
}
\end{lstlisting}

The other methods should be self-explanatory from the comments above. Let course staff know if you have any confusion about these. Make sure to use the provided main method to test your code. We are not providing tests for this assignment.

%------------------------------------------------

\subsection{Gradescope}

You should submit your code to \emph{Gradescope}. If you are having trouble with your submission, you should double check the following common problems:

\begin{enumerate}
	\item Make sure you are only submitting one file, and it is called \emph{BigOh.java} exactly.
	\item Make sure there is keep the package statement at the top of the file (package main;). You can remove the one that is given in the default file that is provided when you submit.
	\item Your code should \textbf{NOT contain any additional import statements} \emph{*the import of Vector can remain}.
	\item Make sure you \textbf{DO NOT} submit a \emph{main method} in your file.
	\item Make sure your file is \textbf{NOT printing anything to the console}. Your methods should simply return the correct results.
\end{enumerate}

%----------------------------------------------------------------------------------------

\end{document}


%----------------------------------------------------------------------------------------
%----------------------------------------------------------------------------------------
%----------------------------------------------------------------------------------------
%----------------------------------------------------------------------------------------
%----------------------------------------------------------------------------------------
%----------------------------------------------------------------------------------------


%WORKS CITED:

%%%%%%%%%%%%%%%%%%%%%%%%%%%%%%%%%%%%%%%%%
% Short Sectioned Assignment
% LaTeX Template
% Version 1.0 (5/5/12)
%
% This template has been downloaded from:
% http://www.LaTeXTemplates.com
%
% Original author:
% Frits Wenneker (http://www.howtotex.com)
%
% License:
% CC BY-NC-SA 3.0 (http://creativecommons.org/licenses/by-nc-sa/3.0/)
%
%%%%%%%%%%%%%%%%%%%%%%%%%%%%%%%%%%%%%%%%%

%----------------------------------------------------------------------------------------
%	PACKAGES AND OTHER DOCUMENT CONFIGURATIONS
%----------------------------------------------------------------------------------------
